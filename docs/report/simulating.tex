\section{Simulation}
Für die Simulation gibt es unterschiedliche Strategien, welche das Verhalten des Lagers beinflussen. Neben den Teilen für das Hochregallager selber, gibt es auch Punkte, die nur das Regalbediengerät berücksichtigen.

\subsection{Strategien}
\begin{itemize}
 \item Bewegungsstrategien
 \begin{itemize}
  \item Einlagerung
  \item Auslagerung
  \item kombiniert
\end{itemize}
 \item Ruhepositionsstrategien
 \begin{itemize}
  \item Verweilen am letzten Arbeitspunkt
  \item Rückkehr zur Übergabestelle
  \item Freie Posistion in der Regalgasse
\end{itemize}
 \item Einlagerugnsstrategien
 \begin{itemize}
  \item zufällige Einlagerung
  \item Einlagerung nahe der Auslagerung
  \item chaotische Einlagerung
  \item zonierte Einlagerung
\end{itemize}
 \item Auslagerugnsstrategien
 \begin{itemize}
  \item strenges FIFO
  \item abgeschächtes FIFO
\end{itemize}
 \item Umlagerungsstrategien
 \begin{itemize}
  \item keine Umlagerungen
  \item zufällige Umlagerungen
  \item zonierte Umlagerung
\end{itemize}
 \item Reihenfolgestrategien
 \begin{itemize}
  \item First come, first serve
  \item Fahrweg- und Zeit-Optimierung
  \item lokale Queue-Optimierung
  \item globale Optimierung
\end{itemize}
 \item Nichtbeschäftigungsstrategien
\end{itemize}




%EOF
