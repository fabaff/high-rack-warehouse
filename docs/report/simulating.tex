\section{Simulation}
\subsection{Allgemein}
Die Simulation wurde als ereignisgesteuerte Echtzeitsimulation implementiert. Die Basis sind Lageraufträge wie Einlagerungs-, Auslagerungs- und Umlagerungsaufträge. Zu Beginn der Simulation muss (momentan) mindestens 1 Auftrag vorhanden sein, da die leere Auftragsliste, bzw. die daraus resultierende leere Ereignisliste, das Abbruchkriterium ist, um die Simulation zu beenden. Dies könnte noch verbessert werden, indem die gewünschte Simulationsendzeit als Erinnerungsereignis in die Ereignisliste eingetragen würde. Die Simulation würde dann nach Abarbeitung dieses letzten Erinnerungsereignisses gestoppt, sofern mittlerweile keine noch späteren Ereignisse eingetragen wurden. \\
Es könnten während der laufenden Simulation weitere Aufträge hinzugefügt werden. Die Startzeit des Auftrags muss allerdings die aktuelle Simulationszeit übersteigen. Bei der Eingabe von zusätzlichen Aufträgen ist das Zeitformat zu berücksichtigen. Diese Funktion ist momentan nicht implementiert.
\\
Für jeden zukünftigen Auftrag wird ein Erinnerungsereignis in die Ereignisliste eingetragen mit Ereigniszeit = Startzeit des Auftrags. Dies dient dazu, dass die Simulation selbst nur die Ereignisliste abarbeiten kann, indem die Simulation jeweils das früheste Ereignis aus der Liste entfernt, anhand der Simulationszeit die benötigte Wartezeit berechnet und dann exakt zur richtigen Simulationszeit das Ereignis ausführt. In der Regel führt jedes Ereignis zu einem Nachfolgeereignis, welches
wiederum in die Ereignisliste eingetragen wird. Dieser Vorgang wird solange wiederholt, bis die Ereignisliste leer ist, wodurch die Simulation dadurch
beendet wird.
%
\subsection{Spezialfälle}
Einzig das letzte Ereignis eines Auftrags sowie unter Umständen ein Erinnerungsereignis können dazuführen, dass kein Nachfolgeereignis mehr erstellt wird. Dies ist der Fall, wenn für die aktuelle Gasse keine weiteren Aufträge mehr vorhanden sind oder überhaupt keine Aufträge mehr zur Abarbeitung bereitstehen.\\
Würden keine Erinnerungsereignisse generiert, sondern direkt das erste Ereignis des entsprechenden Auftrags, wäre es schwierig  zusätzliche Aufträge, die vor dem anderen Ereignis startet, in die Ereignisliste einzufügen und diese Liste entsprechend zu bereinigen.\\
Dies wäre aber notwendig, da ansonsten zwei unterschiedliche Aufträge in derselben Gasse parallel ausgeführt würden. Dies ist aber physikalisch unmöglich und darf in der Simulation nicht geschehen! Bei jedem Erinnerungsereignis wird deshalb nur die Auftragsliste auf fällige Aufträge geprüft und bei Bedarf das erste Ereignis für diese Aufträge generiert. Es werden aber nur Ereignisse generiert für Aufträge von Gassen, welche nicht bereits durch andere, momentan ausgeführte Aufträge blockiert sind. Für diese Prüfung wird wiederum die Ereignisliste herangezogen.\\
%
\subsection{Berechnung Simulationszeit}
Die aktuelle Simulationszeit wird anhand folgender Formel berechnet:\\
\begin{equation}
   \begin{split}
 Aktuelle Simulationszeit [ms] = Simulationsstartzeit [ms] \\
  + Faktor * (Aktuelle Systemzeit [ms] – Startsystemzeit [ms])
     \end{split}
\end{equation}
\\
Da im Modus \textit{As Fast As Possible} nicht auf die exakte Ausführungszeit gewartet werden soll, sondern die Ereignisse sofort ausgeführt werden sollen, wird die Formel mit einem Korrekturwert ergänzt, welcher der Summe der Wartezeiten entspricht:
\\%
\begin{equation}
   \begin{split}
Aktuelle Simulationszeit [ms] = Simulationsstartzeit [ms] \\
  + Faktor * (Aktuelle Systemzeit [ms] – Startsystemzeit [ms]) + Korrekturwert [ms]
     \end{split}
  \end{equation}
\\%
Der Korrekturwert bleibt bei der Echtzeitsimulation immer 0, weshalb für beide Simualtionsmodi dieselbe Formel verwendet werden kann.
%
\subsection{Fehler durch Rechenzeit und dessen Kompensation}
Es gibt nun aber Fälle, in denen die Ereignisse anhand der aktuellen Simulationszeit zu spät ausgeführt werden. Dies dadurch, dass mehrere Ereignisse die gleiche oder zumindest eine kurz aufeinanderfolgende Startzeit haben. Der Grund für die Verzögerung ist, dass ja das \textit{ausführen} der Ereignisse Rechenzeit benötigt und die Berechnung der aktuellen Simulationszeit auf der Systemzeit beruht. Diese Verzögerungen würden sich summieren und müssen unbedingt kompensiert werden.\\
Also wird für die Nachfolgeereignisse nicht die aktuelle Simulationszeit als Basis zur Berechnung der Startzeit des Nachfolgeereignisses verwendet, sondern die Startzeit seines Vorgängerereignisses. Die aktuelle Simulationszeit wird also nur verwendet, um die Wartezeit bis zur Ausführung des nächsten Ereignisse zu berechnen und um zu jedem Zeitpunkt der Simulation den entsprechenden Zustand bzw. die aktuellen Koordinaten der Regalbediengeräte zu berechnen
%
\subsection{Mögliche Strategien}
Für die Simulation gibt es unterschiedliche Strategien, welche das Verhalten des Lagers beinflussen. Neben den Teilen für das Hochregallager selber, gäbe es auch Punkte, die nur das Regalbediengerät berücksichtigen.
%
% \begin{itemize}
%  \item Bewegungsstrategien
%  \begin{itemize}
%   \item Einlagerung
%   \item Auslagerung
%   \item kombiniert
% \end{itemize}
%  \item Ruhepositionsstrategien
%  \begin{itemize}
%   \item Verweilen am letzten Arbeitspunkt
%   \item Rückkehr zur Übergabestelle
%   \item Freie Posistion in der Regalgasse
% \end{itemize}
%  \item Einlagerugnsstrategien
%  \begin{itemize}
%   \item zufällige Einlagerung
%   \item Einlagerung nahe der Auslagerung
%   \item chaotische Einlagerung
%   \item zonierte Einlagerung
% \end{itemize}
%  \item Auslagerugnsstrategien
%  \begin{itemize}
%   \item strenges FIFO
%   \item abgeschwächtes FIFO
% \end{itemize}
%  \item Umlagerungsstrategien
%  \begin{itemize}
%   \item keine Umlagerungen
%   \item zufällige Umlagerungen
%   \item zonierte Umlagerung
% \end{itemize}
%  \item Reihenfolgestrategien
%  \begin{itemize}
%   \item First come, first serve
%   \item Fahrweg- und Zeit-Optimierung
%   \item lokale Queue-Optimierung
%   \item globale Optimierung
% \end{itemize}
%  \item Nichtbeschäftigungsstrategien
% \end{itemize}
%
Im momentan Stand wird das Regalbediengerät, als Ruhepositionsstrategien, immer wieder auf die Warteposition bei der Vorzone gesetzt. Ohne statistische Daten einer Lagerort-Konfiguration kann die bestmögliche Strategie für die Ruheposition nicht bestimmt werden.\\
%
Das Hochregallager wird mit den Artikel aus der Artikelliste (siehe \ref{art-list} auf Seite \pageref{art-list}) per Zufall gefüllt. Eine Überprüfung stellt sicher, dass die ersten Aufträge erledigt werden können. Das heisst, dass für die Auslagerung die Fächer belegt sind und bei der Einlagerung die Fächer frei.
%
\subsection{Ausführen der Simulation}
Die Simulation lässt sich durch Kommandozeilen-Argumente steuern. 

\begin{verbatim}
Arguments definition:
-m (afap | f) // as fast as possible, factor
-f (double)   // factor 
-l (int)      // location number, joblist number, itemlist number
-p (int)      // print type, 0 for file, any other for console 
\end{verbatim}

As fast as possible, Faktor 1 (wird überschrieben), Lagerort 2 laden, Ausgabe in Datei
\begin{verbatim}
-m afap -f 2.5 -l 2 -p 0
\end{verbatim}
As fast as possible, Faktor 1, Lagerort 3 laden, Ausgabe in Konsole
\begin{verbatim}
-m afap -f 1 -l 3 -p 15
\end{verbatim}
Mit Faktor, Faktor 2.5, Lagerort 2 laden, Ausgabe in Konsole
\begin{verbatim}
-m f -f 2.5 -l 2 -p 14
\end{verbatim}
Mit Faktor, Faktor 1, Lagerort 3 laden, Ausgabe in Datei
\begin{verbatim}
-m f -f 1 -l 3 -p 0			  
\end{verbatim}
%
\subsection{Erzeugung der Eingabedaten}
Für die Simulation werden flache Text-Dateien verwendet, welche die relevanten Daten in einer Strichpunkt-getrennten Schreibweise enthalten. Es werden eine Definition des Lagerorts, einer Liste mit Artikeln und einer Auftragsiste. 
%
\paragraph{Lagerort-Liste (locationX.txt)}
\begin{verbatim}
Location;my location;3;MM
Gap;Gasse1;1000;Grid1;Grid2;7;4
Grid;Grid1;1000
Grid;Grid2;1000
Column;A;1200
...
Row;1;600
...
Gap;Gasse2;1000;Grid3;Grid4;5;3
Grid;Grid3;1000
...
Column;A;1000
...
Row;1;500
...
Gap;Gasse3;600;Grid5;Grid6;4;4
Grid;Grid5;600
Grid;Grid6;600
Column;A;1000
...
Row;1;500
...
...
\end{verbatim}
%
\paragraph{Artikel-Liste (item\_listX.txt)}\label{art-list}
\begin{verbatim}
00001;Article 1
00002;Article 2
00003;Article 3
00004;Article 4
00005;Article 5
00006;Article 6
...
\end{verbatim}
%
\paragraph{Auftragsliste (job\_listX.txt)}
\begin{verbatim}
2000.01.01 00:00:00.300;O;Gasse1-1-Grid2-E-3
2000.01.01 00:00:00.200.123;I;Gasse3-1-Grid6-C-2;00123
2000.01.01 00:00:00.100;I;Gasse3-1-Grid6-C-4;00001
2000.01.01 00:00:00.400;O;Gasse2-0-Grid3-E-2
...
\end{verbatim}
%



%EOF
