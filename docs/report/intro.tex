\section{Einleitung}
Ein Hochregallager (HRL) beschreibt ein Lagersystem mit Plätzen in sogenannten Regalen. Hochregallager gibt ein in den unterschiedlichsten Ausprägungen. Die grössten Ausführungen besitzen Höhen bis etwa 50 m und können mehreren hunderttausend Plätze besitzen. Oftmals werden direkt Euro-Paletten als Träger für das Lagergut verwendet, ist das Lagergut zu klein, werden häufig spezielle Kunststoff-Behälter benutzt.\\
Grobgesagt besteht ein Hochregallager aus einer bestimmten Anzahl von Gassen. Eine Gasse wiederum hat links und rechts Lagerplätze und im Freiraum bewegt sich ein Bediengerät. In einem manuellen Hochregallager ist dieser Raum so gross, dass mit einem Gabelstapler zwischen den Regalwänden manövriert werden kann. Bei automatischen Lagern fährt ein Bediengerät, welches von einem Lagerverwaltungssystem seine Befehle bekommt, ohne manuelle Interventionen in der Gasse und liefert das Lagergut zur Entnahmestelle.\\
Die Hochregallager haben eine hohe Raumnutzung und bei der Erstellung sind hohe Investitionen nötig, da bei kleiner Ausführungen eine Halle um das Hochregallager gebaut werden muss. Bei grossen Varianten wird das Hochregal als Tragstruktur für das Gebäude mitbenutzt.\\
Dieses Dokument beschreibt in stark reduzierter Form die wichtigsten Aspekte der Anwendung. Die Installations- und Benutzungsanleitung ist separat verfügbar.
%
\subsection{Rahmenbedingungen}
Die verwendete Programmiersprache ist Java. Die Abgabe des Projektes ist in der letzten Woche des Herbstsemester 2013/2014.
%
\subsection{Abgrenzung}
Die Schnittstelle liegt an der Stirnseite des Hochregallagers zur Vorzone. Die gestrichelte Linie in \ref{fig:overview} auf Seite \pageref{fig:overview} stellt diese Grenze dar. Das Hochregal hat keine fest definierte Abmessungen und auch die Lagergüter habe keine definierten Masse oder Gewicht. Es wird davon ausgegangen, dass die Lagergüter auf einem Träger (z. B. Euro-Palette) platziert sind oder sich in einem Behälter befinden und so zum Hochregal kommen. Alle Lagergüter werden mit den gleichen physikalischen Bedingungen befördert.\\ 
%EOF
