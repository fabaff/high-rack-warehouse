\section{Resultate}
Die Ausgabe des Durchlaufs wird in der Konsole ausgeben. Es sind die folgenden Informationen vorhanden:
%
\begin{itemize}
  \item Kompletter Lagerort
  \item Lagerplätze und deren Belegung
  \item Bekannte Jobs beim Simulationsstart
  \item Gewählte Parameter
  \item Neuer Zustand des gesamten Lagerorts
\end{itemize}

\begin{verbatim}
Lagerplatzzuweisung vor Start der Simulation:
---------------------------------------------
Lagerort:
  Name = my location 2
  Anzahl Gassen: 5
    Name: Gasse1, X-Koordinate: 1000, Breite: 1000
    Grid L: Name: Grid1, Tiefe: 1000, Laenge: 20400, Hoehe: 11250
    Grid R: Name: Grid2, Tiefe: 1000, Laenge: 20400, Hoehe: 11250

    Name: Gasse2, X-Koordinate: 4000, Breite: 1000
    Grid L: Name: Grid3, Tiefe: 1000, Laenge: 4800, Hoehe: 1600
    Grid R: Name: Grid4, Tiefe: 1000, Laenge: 4800, Hoehe: 1600

 [snip]

  Anzahl Bin: 710

  1 - ID: Gasse1-0-Grid1-A-1, Artikel:            
          , Koordinaten: (X/Y/Z/U) : 1000/0/0/-1000
  2 - ID: Gasse1-0-Grid1-A-2, Artikel:            
          , Koordinaten: (X/Y/Z/U) : 1000/0/600/-1000
  3 - ID: Gasse1-0-Grid1-A-3, Artikel:            
          , Koordinaten: (X/Y/Z/U) : 1000/0/1400/-1000
  4 - ID: Gasse1-0-Grid1-A-4, Artikel:            
          , Koordinaten: (X/Y/Z/U) : 1000/0/1900/-1000
  5 - ID: Gasse1-0-Grid1-A-5, Artikel:            
          , Koordinaten: (X/Y/Z/U) : 1000/0/2500/-1000
  6 - ID: Gasse1-0-Grid1-A-6, Artikel: Article 828
          , Koordinaten: (X/Y/Z/U) : 1000/0/3250/-1000
  7 - ID: Gasse1-0-Grid1-A-7, Artikel:            
          , Koordinaten: (X/Y/Z/U) : 1000/0/4050/-1000
  8 - ID: Gasse1-0-Grid1-A-8, Artikel: Article 447
          , Koordinaten: (X/Y/Z/U) : 1000/0/4550/-1000
 [snip]
709 - ID: Gasse5-1-Grid10-D-3, Artikel:           
          , Koordinaten: (X/Y/Z/U) : 14000/3200/1000/1000
710 - ID: Gasse5-1-Grid10-D-4, Artikel: Article 1382
          , Koordinaten: (X/Y/Z/U) : 14000/3200/1600/1000

Bereits bekannte Jobs beim Start der Simulation:
------------------------------------------------
Job: ID = '  4', RackFeeder = 'Gasse3', Startzeit = 2000.01.01 00:00:00.049
Job: ID = '  2', RackFeeder = 'Gasse2', Startzeit = 2000.01.01 00:00:00.050
Job: ID = '  3', RackFeeder = 'Gasse3', Startzeit = 2000.01.01 00:00:02.321
[snip]

Erinnerungsevent hinzugefuegt; Eventzeit: 2000.01.01 00:00:00.049
Erinnerungsevent hinzugefuegt; Eventzeit: 2000.01.01 00:00:00.050
[snip]

Simulation wird nun gestartet:
Aktuelle Systemzeit: 2014.01.17 12:48:50.657
Aktueller Faktor: 1.0
Aktueller Modus: AS_FAST_AS_POSSIBLE

---------------------------------------------------
1        -------------------------------------------------
Simulationszeit: 2000:01:01 00:00:00.003
Erinnerungsevent gefunden, Startzeit: 2000.01.01 00:00:00.049

Wartezeit in ms (simuliert): 45

Simulationszeit (echt) nach Wartezeit: 2000:01:01 00:00:00.050
Simulationszeit (soll) nach Wartezeit: 2000.01.01 00:00:00.049

Kein Nachfolgeevent. Eventuell Erinnerungsevents oder Startevents anlegen?
Event hinzugefuegt fuer Job '4', RackFeeder 'Gasse3'; 
  Eventzeit:  2000.01.01 00:00:00.049
-------------------------------------------------
2        ------------------------------------------
Simulationszeit: 2000:01:01 00:00:00.079
[snip]
Wartezeit in ms (simuliert): 0

Simulationszeit (echt) nach Wartezeit: 2000:01:01 00:10:07.744
Simulationszeit (soll) nach Wartezeit: 2000.01.01 00:10:07.740

Kein Nachfolgeevent. Eventuell Erinnerungsevents oder Startevents anlegen?
----------------------------------------------------
Simulation wird nun beendet, Start-Systemzeit: 2014.01.17 12:48:50.657
                          aktuelle Systemzeit: 2014.01.17 12:48:51.140

Vergangene Systemzeit in Millis: 483
\end{verbatim}
Die komplette Ausgabe kann auch in eine Datei geschrieben werden, was einen späteren Vergleich der Durchgänge ermöglicht.

%EOF
