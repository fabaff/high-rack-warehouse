\documentclass[11pt,a4paper]{article}
\usepackage[paper=a4paper,left=25mm,right=20mm,top=30mm,bottom=30mm]{geometry} 
\usepackage[english, ngerman]{babel}
\usepackage[utf8]{inputenc}
\usepackage[colorlinks=false]{hyperref}
\usepackage[table,usenames,dvipsnames]{xcolor}
\usepackage{tabularx}
\usepackage{graphicx}
\usepackage{listings}
\usepackage{float}
\usepackage{hyperref}
\usepackage{multicol}
\usepackage{fancyhdr}
\usepackage{sectsty}
\usepackage[final]{pdfpages}
\usepackage{threeparttable}
\usepackage{gitinfo}
\usepackage{totpages}
\usepackage{datetime}
\usepackage{subcaption}
\usepackage[printonlyused]{acronym}
\usepackage[toc]{glossaries}

%Definitionen
\includepdfset{pages=-} %Alle Seite importieren  Option: noautoscale
\setlength{\parindent}{0em} %Einrueckung eines neuen Absatzes, 0 keine Einrueckung

\definecolor{rowc1}{RGB}{40,75,95}

\setcounter{tocdepth}{3}
\setcounter{secnumdepth}{3}

\ifpdf
\pdfinfo {
	/Author (Marc Schärer, Arthur van Ommen, Fabian Affolter)
	/Title (Report)
	/Subject ()
	/Keywords ()
	/CreationDate (D:\pdfdate)
}
\fi

\pagestyle{fancy}
\renewcommand{\headrulewidth}{0pt}
\lhead{}%\includegraphics[width=80mm]{logo.png}}
\chead{}
\rhead{}
\lfoot{}
\cfoot{\thepage}
\rfoot{}

\renewcommand{\arraystretch}{1.3}
\lstset{basicstyle=\ttfamily,breaklines=true}

\begin{document}
%%%%%%%%%%%%%%%%%%%%%%%%%%%%%%%%%%%%%%%%%%%%%%%%%
{\huge \textbf{Anforderungs-Dokumentation}} - \textbf{Hochregallager} \\

\section{Vorwort}
%
\subsection{Zielgruppe}
Dieses Dokument beschreibt die Anforderungen an eine Software-Lösung für die Simulation und Optimierung von Hochregallagern im Detail. Es enthält die Aspekte der gesuchten Lösung mit dem Fkous auf die technische Seite. Der Leser sollte ein grundlegendes Verständnis von Logistik und Lagerungssystem haben und mit der in diesem Bereich verwendeten Terminologie vertraut sein. 
%
\subsection{Autoren}
Die Autoren diesen Dokument sind:
%
\begin{itemize}
  \item Marc Schärer \href{mailto:scham36@bfh.ch}{\nolinkurl{scham36@bfh.ch}}
  \item Arthur van Ommen \href{mailto:vanoa1@bfh.ch}{\nolinkurl{vanoa1@bfh.ch}}
  \item Fabian Affolter \href{mailto:affof11@bfh.ch}{\nolinkurl{affof1@bfh.ch}}
\end{itemize}
%
\subsection{Dokument-Versionen}

\begin{table}[h]
  %\caption{}
  %\label{tab:releases}

  \begin{center}
    \begin{tabular}{|c|c|c|}
      \hline
      \textbf{Version} & \textbf{Autor} & \textbf{Bemerkungen} \\
      \hline
      0 & Team & Skelett \\
      \hline
    \end{tabular}
  \end{center}
\end{table}
%
\subsection{Glossar}
tbd
%
\section{Einleitung}
tbd
%
\section{Benutzeranforderungen}
%
\subsection{Funktionale Anforderungen}
\begin{itemize}
  \item Definition des Szenarios (Statische Parameter)
  \item Eingeben der Simulationsparameter (dynamische Parameter)
  \item Simulationssteuerung
  \item Szenarienmanagement
\end{itemize}
%
\subsection{Nichtfunktionale Anforderungen}
\begin{itemize}
  \item Keine unsinnig grossen (langlaufende) Simulationen
  \item Grafische Darstellung während der Simulation (informativ)
  \item Sprache der Applikation ist in Englisch
  \item Ausgabe / Export der Ergebnisse auf Drucker oder als Dokument (z.B. .txt, .csv, .xml usw.)
\end{itemize}

%
\subsubsection{Domainspezifische Anforderungen}
\begin{itemize}
  \item Gefahrengut / Brandschutz
  \item Konformität
  \item Arbeitssicherheit
\end{itemize}
%
\section{System-Architektur}
\begin{itemize}
  \item Clientanwendung
  \item Trennung von Simulation, Auswertung und Visualisierung
\end{itemize}
%
\section{Systemanforderungen}
%
\subsection{Funktionale Anforderungen}
\begin{itemize}
  \item Szenario laden
  \item Szenario simulieren / berechnen
  \item Simuliertes Szenario auswerten / ausgeben
  \item 
  \item 
\end{itemize}
%
\subsection{Nichtfunktionale Anforderungen}
\begin{itemize}
  \item Lauffähig auf Standard-Hardware
  \item nur Standard-Software (JRE, Bibliotheken, etc.)
  \item 
  \item 
\end{itemize}
%
\section{System-Evolution}
N/A
%
\section{Testing}
\begin{itemize}
  \item Unit tests
  \item 
  \item 
  \item 
\end{itemize}
%
\section{Mögliche Szenarien / Use cases}
\begin{itemize}
  \item Maschinenbaufirma im 1-Schichtbetrieb mit Fertigung / Montage / Service --> kurze Zugriffszeiten Tagsüber, freie Ressourcen während der Nacht
  \item Versandhandel im 3-Schichtbetrieb mit Bereitstellung / Konvektionierung --> hoher Lagerdurchsatz, 24h-Zugriff für Ein-/Auslagerung
  \item Gleichzeitiges Ein-/Auslagern, Queue
  \item Mehrere Ein-/Ausgabeplätze pro Gasse (auf Z-Achse)
  \item Mehrere Regalbediengeräte pro Gasse (auf X-Achse, bei mehreren Ein-/Ausgabeplätzen auch auf Z-Achse)
  \item Mehrere Ladearme pro Regalbediengerät (mehrere Vertikal, Horizontal [ohne / mit Durchreichemöglichkeit], Radial)
  \item Vorgezogenes Auslagern (Bereitstellung noch im Lager)
  \item Aufall einer Gasse, Mehrplatzeinlagerung gleicher Teile in unterschiedlichen Gassen
  \item Fixe Lagerplatzzuordnung (Reservation, defekte Lagerplätze, abgeschottete Lagerplätze für Gefahrengut)
  \item Befüllung (inital von leerem Lager / Nachbefüllung) (zufällig chaotisch, zeitoptimiert chaotisch, zugeordnet, Positionsoptimiert [ABC])
  \item Optimierung von bereits belegtem Lager aufgrund Zugriff-History
  \item 
\end{itemize}


%%%%%%%%%%%%%%%%%%%%%%%%%%%%%%%%%%%%%%%%%%%%%%%%%%%
\end{document}
