\documentclass[11pt,a4paper]{article}
\usepackage[paper=a4paper,left=25mm,right=20mm,top=30mm,bottom=30mm]{geometry} 
\usepackage[english, ngerman]{babel}
\usepackage[utf8]{inputenc}
\usepackage[colorlinks=false]{hyperref}
\usepackage[table,usenames,dvipsnames]{xcolor}
\usepackage{tabularx}
\usepackage{graphicx}
\usepackage{listings}
\usepackage{float}
\usepackage{hyperref}
\usepackage{multicol}
\usepackage{fancyhdr}
\usepackage{sectsty}
\usepackage[final]{pdfpages}
%\usepackage{gitinfo}
%\usepackage{totpages}
\usepackage{datetime}
\usepackage{subcaption}
%\usepackage[printonlyused]{acronym}
%\usepackage[toc]{glossaries}

%Definitionen
\includepdfset{pages=-} %Alle Seite importieren  Option: noautoscale
\setlength{\parindent}{0em} %Einrueckung eines neuen Absatzes, 0 keine Einrueckung

\definecolor{rowc1}{RGB}{40,75,95}

\setcounter{tocdepth}{3}
\setcounter{secnumdepth}{3}

\ifpdf
\pdfinfo {
	/Author (Marc Schärer, Arthur van Ommen, Fabian Affolter)
	/Title (Report)
	/Subject ()
	/Keywords ()
	/CreationDate (D:\pdfdate)
}
\fi

\pagestyle{fancy}
\renewcommand{\headrulewidth}{0pt}
\lhead{}%\includegraphics[width=80mm]{logo.png}}
\chead{}
\rhead{}
\lfoot{}
\cfoot{\thepage}
\rfoot{}

\renewcommand{\arraystretch}{1.3}
\lstset{basicstyle=\ttfamily,breaklines=true}

\begin{document}
%%%%%%%%%%%%%%%%%%%%%%%%%%%%%%%%%%%%%%%%%%%%%%%%%
{\huge \textbf{Anforderungs-Dokumentation}} - \textbf{Hochregallager} \\
\tableofcontents

\section{Vorwort}
%
\subsection{Zielgruppe}
Dieses Dokument beschreibt die Anforderungen an eine Software-Lösung für die Simulation und Optimierung von Hochregallagern im Detail. Es enthält die Aspekte der gesuchten Lösung mit dem Fokus, welcher auf die technische Seite gerichtet ist. Der Leser sollte ein grundlegendes Verständnis von Logistik und Lagerungssystem haben und mit der in diesem Bereich verwendeten Terminologie vertraut sein. 
%
\subsection{Autoren}
Die Autoren diesen Dokument sind:
%
\begin{itemize}
  \item Marc Schärer \href{mailto:scham36@bfh.ch}{\nolinkurl{scham36@bfh.ch}}
  \item Arthur van Ommen \href{mailto:vanoa1@bfh.ch}{\nolinkurl{vanoa1@bfh.ch}}
  \item Fabian Affolter \href{mailto:affof11@bfh.ch}{\nolinkurl{affof1@bfh.ch}}
\end{itemize}
%
\subsection{Dokument-Versionen}

\begin{table}[h]
  %\caption{}
  %\label{tab:releases}

  \begin{center}
    \begin{tabular}{|c|c|c|c|}
      \hline
      \textbf{Version} & \textbf{Autor} & \textbf{Bemerkungen} & Datum\\
      \hline
      0 & Team & Skelett & 20.09.2013\\
      0.1 & Team & erste Version, Dokumentation der Anforderungen &  27.09.2013 \\
      0.2 & Team & überarbeitete Version, Prioritätenliste, Schwerpunkte, u. a. & 01.10.2013 \\
      0.3 & Team & überarbeitete Version nach Besprechung & 08.10.2013 \\
      0.4 & Team & überarbeitete Version nach Besprechung & 08.10.2013 \\
      \hline
    \end{tabular}
  \end{center}
\end{table}
%
%\subsection{Glossar}
%tbd
%
\section{Einleitung}
Ein Hochregallager (HRL) beschreibt ein Lagersystem mit Plätzen in sogenannten Regalen. Hochregallager gibt ein in den unterschiedlichsten Ausprägungen. Die grössten Ausführungen besitzen Höhen bis etwa 50 m und können mehreren hunderttausend Plätze besitzen. Oftmals werden direkt Euro-Paletten als Träger für das Lagergut verwendet, ist das Lagergut zu klein, werden häufig spezielle Kunststoff-Behälter benutzt.\\
Grobgesagt besteht ein Hochregallager aus einer bestimmten Anzahl von Gassen. Eine Gasse wiederum hat links und rechts Lagerplätze und im Freiraum bewegt sich ein Bediengerät. In einem manuellen Hochregallager ist dieser Raum so gross, dass mit einem Gabelstapler zwischen den Regalwänden manövriert werden kann. Bei automatischen Lagern fährt ein Bediengerät, welches von einem Lagerverwaltungssystem seine Befehle bekommt, ohne manuelle Interventionen in der Gasse und liefert das Lagergut zur Entnahmestelle.\\
Die Hochregallager haben eine hohe Raumnutzung und bei der Erstellung sind hohe Investitionen nötig, da bei kleiner Ausführungen eine Halle um das Hochregallager gebaut werden muss. Bei grossen Varianten wird das Hochregal als Tragstruktur für das Gebäude mitbenutzt. 
%
\section{Benutzeranforderungen}
%
\subsection{Funktionale Anforderungen}
\begin{itemize}
  \item Definition des Szenarios (statische Parameter)
  \begin{itemize}
    \item Grundkonfigurationen (Beispiel für ein Lager mit 10000 Fächer)
    \begin{itemize}
      \item Anzahl der Lagergassen
      \item Definition der einzelnen Lagergassen (Lagerfächer ein-/beidseitig, Höhe/Breite/Tiefe der Lagergassen und Lagerfächer)
%      \item 
    \end{itemize}
    
    \item Anzahl der Lagergüter (ergibt die benötigte Fächer-Anzahl)
    \item Geometrische Bedingungen (maximale Gebäude-Abmessungen oder ähnlich)
%    \item 
  \end{itemize}
  
  \item Eingeben der Simulationsparameter (dynamische Parameter)
    \begin{itemize}
    \item Maximale Masse der Lagergüter
    \item Physikalische Eigenschaften der Lagergüter
    \item Geschwindigkeit der beweglichen Elemente (RBG)
    \item Beschleunigungs- und Verzögerungsverhalten der beweglichen Elemente (RBG)
%    \item 
  \end{itemize}
  \item Simulationssteuerung
    \begin{itemize}
    \item Diverse Modi (zeitliche Intervalle, so schnell wie möglich)
    \item 
%    \item 
%    \item 
  \end{itemize}
  \item Szenarienmanagement
    \begin{itemize}
    \item Laden von vordefinierten Szenarien (gemäss Auflistung in Abschnitt \ref{szenarien} auf Seite \pageref{szenarien})
    \item Laden von eigenen Szenarien
    \item Speichern von erstellten Szenarien
%    \item 
  \end{itemize}
\end{itemize}
%
\subsection{Nichtfunktionale Anforderungen}
\begin{itemize}
  \item Keine unsinnig grossen (langlaufende) Simulationen
  \item Grafische Darstellung während der Simulation (informativ)
  \item Sprache der Applikation ist in Englisch
  \item Ausgabe / Export der Ergebnisse auf Drucker oder als Dokument (z.B. .txt, .csv, .xml, usw.)
\end{itemize}

%
\subsubsection{Domainspezifische Anforderungen}
\begin{itemize}
  \item Gefahrengut / Brandschutz
  \item Konformität
  \item Arbeitssicherheit
\end{itemize}
%
\section{System-Architektur}
\begin{itemize}
  \item Die Anwendung soll eine reine Clientanwendung sein
  \item Trennung von Simulation, Auswertung und Visualisierung: die Simulation soll unabhängig einer gewählten Visualisierung ablaufen. D.h. keine für die Simulation benötigte Logik oder Komponenten innerhalb der Visualisierung. Ebenfalls soll die Simulation unabhängig der gewünschten Auswertungsart ablaufen können.
\end{itemize}
%
\section{Systemanforderungen}
%
\subsection{Funktionale Anforderungen}
\begin{itemize}
  \item Szenario laden
  \item Szenario simulieren / berechnen
  \item Simuliertes Szenario auswerten / ausgeben
%  \item 
%  \item 
\end{itemize}
%
\subsection{Nichtfunktionale Anforderungen}
\begin{itemize}
  \item Lauffähig auf Standard-Hardware
  \item Nur Standard-Software (JRE, Bibliotheken, etc.)
%  \item 
%  \item 
\end{itemize}
%
\section{System-Evolution}
N/A
%
\section{Testing}
\begin{itemize}
  \item Unit tests
%  \item 
%  \item 
%  \item 
\end{itemize}
%
\section{Mögliche Szenarien}\label{szenarien}
Dieser Abschnitt beschreibt mögliche Szenarien, welche in Simulationen betrachtet werden könnten.
%
\begin{itemize}
  \item Maschinenbaufirma im 1-Schichtbetrieb mit Fertigung / Montage / Service -- kurze Zugriffszeiten Tagsüber, freie Ressourcen während der Nacht
  \item Versandhandel im 3-Schichtbetrieb mit Bereitstellung / Konvektionierung --  hoher Lagerdurchsatz, 24h-Zugriff für Ein-/Auslagerung
  \item Gleichzeitiges Ein-/Auslagern, Queue
  \item Mehrere Ein-/Ausgabeplätze pro Gasse (auf z-Achse)
  \item Mehrere Regalbediengeräte pro Gasse (auf x-Achse, bei mehreren Ein-/Ausgabeplätzen auch auf z-Achse)
  \item Mehrere Ladearme pro Regalbediengerät (mehrere vertikal, horizontal [ohne / mit Durchreichemöglichkeit], radial)
  \item Vorgezogenes Auslagern (Bereitstellung noch im Lager)
  \item Ausfall einer Gasse, Mehrplatzeinlagerung gleicher Teile in unterschiedlichen Gassen
  \item Fixe Lagerplatzzuordnung (Reservation, defekte Lagerplätze, abgeschottete Lagerplätze für Gefahrengut)
  \item Befüllung (inital von leerem Lager / Nachbefüllung) (zufällig chaotisch, zeitoptimiert chaotisch, zugeordnet, positionsoptimiert [ABC])
  \item Optimierung von bereits belegtem Lager aufgrund Zugriff-History
%  \item 
\end{itemize}
%
\section{Schwerpunkte}
\begin{itemize}
  \item Simulation / Simulationsauswertung / Simulationsvisualisierung
  \item Optimierung
\end{itemize}
%
\section{Möglicher Ablauf der Arbeitsschritte}

\begin{enumerate}
  \item Laden eines einfachen Szenarios (mit fixen Einstellungen)
  \item Berechnung der Simulations-Eckdaten
  \item Visuelle Darstellung des Szenario (2D) -- Überprüfung des Klassen-Diagramm, Abschätzung der Performance
  \item Steigerung der Komplexität der Szenarien (einfache Input-Funktion, ASCII- oder Spreadsheet-Datei) -- Szenarien-Management
  \item Simulierte Szenarien auswerten / einfache Daten-Ausgabe
  \item Optimierung von einzelnen Szenarien (nach einer vorgegebenen Auswertung, nach vorgebener Strategie, Änderung der Hochregallager-Parameter)
  \item Erweiterung der Export-Funktion (Spreadsheet oder ähnlich, für grafische Auswertungen)
  \item Erweiterung der visuellen Repräsentation (3D)
  \item Ausgabe für die Dimensionierung/Auslegung von Hochregallagern
  \item Einbezug der Vorzone in Simulation/Optimierung
  \item Multifunktionale Benutzeroberfläche für die Eingabe/Simulation/Auswertung/Auslegung
\end{enumerate}
%
\section{Sonstiges}
\begin{itemize}
  \item Repository: \href{https://github.com/fabaff/high-rack-warehouse}{https://github.com/fabaff/high-rack-warehouse}
  \item Dokumentation: \href{https://github.com/fabaff/high-rack-warehouse/tree/master/docs}{/docs}
  \item Code: \href{https://github.com/fabaff/high-rack-warehouse/}{Pfad momentan unbestimmt}
\end{itemize}
%%%%%%%%%%%%%%%%%%%%%%%%%%%%%%%%%%%%%%%%%%%%%%%%%%%
\end{document}
